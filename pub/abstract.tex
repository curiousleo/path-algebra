Dijkstra's Algorithm is typically presented as operating on graphs with numeric arc weights, but a more general form of the algorithm exists that operates on graphs where arc weights are drawn from a large class of semirings.
In this more general setting, standard correctness proofs rely on the distributivity property of semirings to show that Dijkstra's Algorithm computes the best path weights over all possible paths from a single source node to all potential destinations, a process which can be interpreted as finding a fixed point for a certain matrix equation.
In this paper we present a mechanised proof that Dijkstra's Algorithm can solve these matrix equations even when distributivity does not hold.
Fixed points to matrix equations for non-distributive algebras can be interpreted as representing `locally optimal' solutions.
We use Agda for our implementation and proof, making use of dependent types and some of Agda's more cutting edge features---such as induction-recursion---to structure our algorithm and correctness proof.

\keywords{Dijkstra's Algorithm, semirings, interactive theorem proving}
