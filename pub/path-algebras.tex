\begin{figure}[t]
\centering
\begin{tabular}{c||l@{\;\;\;}|l}
\textbf{Operation} & \textbf{Semiring} & \textbf{Sobrinho Algebra} \\
\midrule
\AgdaFunction{\_+\_} & Associative & Associative \\
                 & Commutative & Commutative \\
                 & Identity: \AgdaField{0\#} & Identity: \AgdaField{0\#} \\
                 & ---                      & Selective \\
                 & ---                      & Zero: \AgdaField{1\#} \\
\midrule
\AgdaFunction{\_*\_} & Associative & --- \\
                 & Identity: \AgdaField{1\#} & Left identity: \AgdaField{1\#} \\
                 & Zero: \AgdaField{0\#}     & --- \\
\midrule
\AgdaFunction{\_*\_} and \AgdaFunction{\_+\_} & \AgdaFunction{\_*\_} distributes over \AgdaFunction{\_+\_} &
                   \AgdaFunction{\_+\_} absorbs \AgdaFunction{\_*\_} \\
\bottomrule
\end{tabular}
\label{tab.path.algebra}
\vspace{6pt}
\caption{Comparing the algebraic properties of a Semiring and a Sobrinho Algebra.}
\label{fig.path.algebra}
\end{figure}

Fix a set $S$ and an equivalence relation $- ≈ -$.
Call a binary operation on $S$, $- \bullet -$, \emph{selective} when for all $x, y \in S$ either $x \bullet y ≈ x$ or $x \bullet y ≈ y$.
With this definition in mind, we call a structure $\langle S, +, *, 0, 1 \rangle$ a `Sobrinho Algebra' when:
\begin{itemize}
\item
$\langle S, +, 0 \rangle$ forms a commutative monoid,
\item
$1$ is a left identity for multiplication, and a left- and right zero for addition,
\item
addition is selective, and addition absorbs multiplication,
\item
the usual closure properties for the unit elements and operations apply.
\end{itemize}
A comparison between Sobrinho Algebras and the more familiar notion of Semiring is presented in Figure~\ref{fig.path.algebra}.

Following established convention, we capture the notion of a Sobrinho Algebra as an Agda record named \AgdaRecord{SobrinhoAlgebra}.
We call the carrier type of a Sobrinho Algebra (the set $S$ in the definition above) \AgdaField{Carrier}, obtaining the closure properties mentioned above for `free' as a side-effect of Agda's typing discipline, and assume that there exists a decidable setoid equivalence relation on elements of this type, \AgdaField{\_≈\_}.

\todo{finish this subsection}
